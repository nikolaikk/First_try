\documentclass{article}
\usepackage[utf8]{inputenc}
\usepackage{amsmath}
\usepackage{graphicx}


\title{Report\\Beam propagation method and absorption boundary condition in 2-D}
\date{21-04-2-2018}

\author{Prepered by: Nicolae Cadin \\Supervised by: Ramon Springer}

\begin{document}
	
	
	\pagenumbering{gobble}
	\maketitle
	\newpage
	\tableofcontents
	\pagenumbering{arabic}
	
	\newpage
	\section{Introduction}
	Beam propagation in media is of high interest for scientists and also for industrial purposes. 
	In this report beam propagation will be simulated, problems will be discussed. The purpose of the project is to compare numerical method and analytical solution of beam propagation. Convergence of used numerical method is to be shown. For the simplicity simulation of beam propagation is done in free space, linear medium, at zero incident angle, so to fit the scope of mini-project, later on can be extended for more sophisticated cases.
	\subsection{Analytical solution}
	To better analytically understand propagation of beam in space we should solve Helmholz Equation.
	\begin{center}
		$(\nabla^2+k^2)E = 0$		
	\end{center}
	This equation generally holds for elegtromagnetic wave, hence for electric and magnetic field components, for simplicity we chose electric field $E$. Here $k$ is wave number and is equal to $2\pi n/\lambda$, where $\lambda$ is wavelength, $n$ is refractive index of medium (here we assume it to be eqaul to 1). As assumption we consider that our wave propagates in $z$ direction and electric field can be represented in next form.
	\begin{center}
		$E(x,y,z)=u(x,y,z)e^{-ik_oz}$
	\end{center}
	$u$ is complex function which describes non-plane part of the beam. Substituting our ansatz into Helmholtz equation in cartesian coordinates and doing some simplifications we get.

	\[\frac{\partial^2 u}{\partial x^2}+ \frac{\partial^2 u}{\partial y^2}+ \frac{\partial^2 u}{\partial z^2} - 2ik_o\frac{\partial u}{\partial z}+(k^2-k_o^2)u=0\]
	Using paraxial approximation we see that third term is smaller than other, therefore can be neglected, as result we get following equation.
	
	\begin{equation}
	\frac{\partial^2 u}{\partial x^2}+ \frac{\partial^2 u}{\partial y^2} - 2ik_o\frac{\partial u}{\partial z}+(k^2-k_o^2)u=0
	\end{equation}
	Solving this differential equation we find $u$ hence we find $E$. We assume that $k = k_o$, therefore solution for one dimensional case, where wave propagates in z-direction and oscillates in xz plane is,
	
	\[E(x,z)=E_o\bigg(\frac{w_o}{w(z)}\bigg)^{\frac{1}{2}}exp\bigg(\frac{x^2}{w(z)^2}\bigg)exp\bigg(-i\Big(k_oz+k_o\frac{x^2}{2R(z)}-\phi(z)\Big)\bigg)\]
	where $w(z)$ is beam radius, $w_o = w(0)$ is beam waist, $E_o$ is electric field amplitude, $R(z)$ is beam curvature, $\phi$ is Gouy phase. Their explicit mathematical formula can be found in the table below.This electric field formula will be later used as reference to numerical solution.
	
%	\[w(z)= w_o\sqrt{1+\Big(\frac{z\lambda}{\pi w_o^2}\Big)^2}\]
%	\[R=z\bigg[1+\Big(\frac{\pi w_o^2}{z\lambda}\Big)^2\bigg]\]
%	\[\phi(z)=arctan\Big(\frac{z\lambda}{\pi w_o^2}\Big)\]

	\begin{table}[h!]
		\begin{center}
%			\caption{Values.}
			\label{tab:table1}
			\begin{tabular}{c| c| c} % <-- Alignments: 1st column left, 2nd middle and 3rd right, with vertical lines in between
				\textbf{Beam Radius} & \textbf{Beam Curvature} & \textbf{Gouy phase}\\
				\hline
				&&\\
				$w(z)= w_o\sqrt{1+\Big(\frac{z\lambda}{\pi w_o^2}\Big)^2}$ & $R=z\bigg[1+\Big(\frac{\pi w_o^2}{z\lambda}\Big)^2\bigg]$ & $\phi(z)=arctan\Big(\frac{z\lambda}{\pi w_o^2}\Big)$\\
			\end{tabular}
		\end{center}
	\end{table}
	
	\subsection{Numerical solution}
	Equation (1) for one dimensional case can be rewritten in the following form.
	\begin{equation}
	2ik_o\frac{\partial u}{\partial z}=\frac{\partial^2 u}{\partial x^2}+(k^2-k_o^2)u
	\end{equation}
	To simulate light beam propagation Beam Propagation Method (BPM) is used. BMP is descritization of formula (2), formula should be descritized for z-component and x-component. For beggining let's do descitization only for z component, our equation will get following form.
	\[2ik_o\frac{u^{l+1}-u^l}{\Delta z}=\frac{\partial^2 u^l}{\partial x^2}+(k^2-k_o^2)u^l\]
	Index $l$ denotes the order of grid point in z-direction. This finite difference scheme is known as {\bf forward difference}. However after discretization of x-component numerical instability can be seen. There is another alternative where order of grid point is taken one step in advance, this method is called {\bf backward difference}
	\[2ik_o\frac{u^{l+1}-u^l}{\Delta z}=\frac{\partial^2 u^{l+1}}{\partial x^2}+(k^2-k_o^2)u^{l+1}\]
	Neither this scheme shows good stability, however the combination of these two solves instability problem. These method is called {\bf Crank-Nicolson} method.
	\[2ik_o\frac{u^{l+1}-u^l}{\Delta z}=(1-\alpha)\frac{\partial^2 u^l}{\partial x^2}+(1-\alpha)(k^2-k_o^2)u^l+\alpha \frac{\partial^2 u^{l+1}}{\partial x^2}+\alpha(k^2-k_o^2)u^{l+1}\]
	$\alpha$ here shows interest of which difference we want higher contribution, usually is set to 1/2. Then Crank-Nicolson scheme has next form.
	\[2ik_o\frac{u^{l+1}-u^l}{\Delta z}=\frac{1}{2}\bigg(\frac{\partial^2 u^l}{\partial x^2}+(k^2-k_o^2)u^l\bigg)+\frac{1}{2}\bigg(\frac{\partial^2 u^{l+1}}{\partial x^2}+(k^2-k_o^2)u^{l+1}\bigg)\]
	Now discitization of x-component can be performed.

	\begin{equation*}
	\begin{split}
	2ik_o\frac{u_j^{l+1}-u_j^l}{\Delta z}=\frac{1}{2}\bigg(\frac{u_{j-1}^l-2u_j^l+u_{j+1}^l}{\Delta x^2}+&(k^2-k_o^2)u_j^l\bigg)+\\
	& \frac{1}{2}\bigg(\frac{u_{j-1}^{l+1}-2u_j^{l+1}+u_{j+1}^{l+1}}{\Delta x^2}+(k^2-k_o^2)u_j^{l+1}\bigg)
	\end{split}
	\end{equation*}
	$j$ index denotes the order of grid point in x-direction. Such discritization can be rewritten and later solved using triagonal matrix algorithm or so called {\bf Thomas algorithm}. 
	\[2ik_o\frac{u^{l+1}-u^l}{\Delta z}=\frac{1}{2}\bigg(L_hu^{l+1}+L_hu^l\bigg)\]
	$L_h$ is spacial discretization operator for one dimensional case and is numerically defined in Thomas algorithm as 

	\setlength\arraycolsep{-2.5pt}
	\[ L = \begin{bmatrix}
    -2+(k^2-k_o^2)\Delta x^2& 1& 0& &\dots& & 0 \\
    1 & -2+(k^2-k_o^2)\Delta x^2 & 1& &\dots& & 0 \\
    0 &     1& -2+(k^2-k_o^2)\Delta x^2 & &\dots& & 0 \\
    \vdots & \vdots & \vdots & &\dots& & \vdots \\
    0& 0& 0& &\dots& &     -2+(k^2-k_o^2)\Delta x^2
	\end{bmatrix}\]
	
	
	From here doing simple mathematical rearrangement we can write previous equation in the following form.
		\[u^{l+1} = \bigg(I-\frac{\Delta z}{4ik_o}L_h\bigg)^{-1}\bigg(I+\frac{\Delta z}{4ik_o}L_h\bigg) u^l\]
	Here $I$ is identity matrix, which has the same size as triagonal matrix $L$. We can notice that this is iterative process, in our particular case we instantiate $u^{l=0}$ at $z = 0$ to Gassuian function, in another words $u(x,z=0)= E_oexp\big(-\frac{x^2}{w_o^2}\big)$.

	\newpage
	\section{Results}
	Results see Figure \ref{fig:Results}
	\begin{figure}[h!]
		\hspace{-30mm}
		\includegraphics[width=1.5\textwidth]{N2.jpg}
		\caption{\label{fig:Results}Here you can see the similarity between two methods.}
%		\centering
	\end{figure}
	\section{Conlusion}
	\end{document}




