\documentclass{article}
\usepackage[utf8]{inputenc}


\title{Report\\Beam propagation method and absorption boundary condition in 2-D}
\date{21-04-2-2018}

\author{Prepered by: Nicolae Cadin \\Supervised by: Ramon Springer}

\begin{document}
	
	
	\pagenumbering{gobble}
	\maketitle
	\newpage
	\tableofcontents
	\pagenumbering{arabic}
	
	\newpage
	\section{Introduction}
	Beam propagation in media is of high interest for scientists and also for industrial purposes. 
	In this report beam propagation will be simulated, problems will be discussed. The purpose of the project is to compare numerical method and analytical solution of beam propagation. Convergence of used numerical method is to be shown. For the simplicity simulation of beam propagation is done in free space, linear medium, at zero incident angle, so to fit the scope of mini-project, later on can be extended for more sophisticated cases.
	\subsection{Analytical solution}
	To better analytically understand propagation of beam in space we should solve Helmholz Eqaution.
	\begin{center}
		$(\nabla^2+k^2)E = 0$		
	\end{center}
	This equation generally holds for elegtromagnetic wave, hence for electric and magnetic field components, for simplicity we chose electric field $E$. Here $k$ is wave number and is equal to $2\pi n/\lambda$, where $\lambda$ is wavelength, $n$ is refractive index of medium (here we assume it to be eqaul to 1). As assumption we consider that our wave propagates in $z$ direction and electric field can be represented in next form.
	\begin{center}
		$E(x,y,z)=u(x,y,z)e^{-ik_oz}$
	\end{center}
	$u$ is complex fuction which describes non-plane part of the beam. Substituting our ansatz into Helmholtz equation in cartesian coordinates and doing some simplifications we get.

	\[\frac{\partial^2 u}{\partial x^2}+ \frac{\partial^2 u}{\partial y^2}+ \frac{\partial^2 u}{\partial z^2} - 2ik\frac{\partial u}{\partial z}+(k^2-k_o^2)u=0\]
	Using paraxial aproximation we see that third term is smaller than other, thefore can be neglected, as result we get following equation.
	
	\begin{equation}
	\frac{\partial^2 u}{\partial x^2}+ \frac{\partial^2 u}{\partial y^2} - 2ik\frac{\partial u}{\partial z}+(k^2-k_o^2)u=0
	\end{equation}
	Solving this differential equation we find $u$ hence we find $E$. Solution for one dimentional case, where wave propagates only in x-z plane is,
	
	\[E(x,z)=E_o\bigg(\frac{w_o}{w(z)}\bigg)^{\frac{1}{2}}exp\bigg(\frac{x^2}{w(z)^2}\bigg)exp\bigg(-i\Big(kz+k\frac{x^2}{2R(z)}-\phi(z)\Big)\bigg)\]
	where
	$w(z)$ is beam radius, $w_o = w(0)$ is beam waist, $E_o$ is electric field amplitude, $R(z)$ is beam curvature, $\phi$ is Gouy phase. Their explicit mathematical formula can be found in the table below.This electric field formula will be later used as reference to numerical solution.
	
%	\[w(z)= w_o\sqrt{1+\Big(\frac{z\lambda}{\pi w_o^2}\Big)^2}\]
%	\[R=z\bigg[1+\Big(\frac{\pi w_o^2}{z\lambda}\Big)^2\bigg]\]
%	\[\phi(z)=arctan\Big(\frac{z\lambda}{\pi w_o^2}\Big)\]

	\begin{table}[h!]
		\begin{center}
%			\caption{Values.}
			\label{tab:table1}
			\begin{tabular}{c| c| c} % <-- Alignments: 1st column left, 2nd middle and 3rd right, with vertical lines in between
				\textbf{Beam Radius} & \textbf{Beam Curvature} & \textbf{Gouy phase}\\
				\hline
				&&\\
				$w(z)= w_o\sqrt{1+\Big(\frac{z\lambda}{\pi w_o^2}\Big)^2}$ & $R=z\bigg[1+\Big(\frac{\pi w_o^2}{z\lambda}\Big)^2\bigg]$ & $\phi(z)=arctan\Big(\frac{z\lambda}{\pi w_o^2}\Big)$\\
			\end{tabular}
		\end{center}
	\end{table}
	
	\subsection{Numerical solution}
	To simulate light beam propagation Beam Propagation Method (BPM) is used. BMP is descritiztion of formula (1), and can be rewritten in following way.
	\[2ik_o\frac{u^{l+1}-u^l}{\Delta z}=\frac{1}{2}\bigg(L_hu^{l+1}+L_hu^l\bigg)\]
	where $L_h$ is spacial discretization operator for one dimensional case is equal to
	\[L= \frac{\partial^2 }{\partial x^2}+(k^2-k_o^2)\]
	L numerically will be defined by Thomas algorithm, also known as triagonal matrix algorithm.
	\section{Algorithm}
	\section{Results}
\end{document}